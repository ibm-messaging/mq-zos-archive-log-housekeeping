%
%* Copyright (c) 2016 IBM Corporation and other Contributors.
%*
%* All rights reserved. This program and the accompanying materials
%* are made available under the terms of the Eclipse Public License v1.0
%* which accompanies this distribution, and is available at
%* http://www.eclipse.org/legal/epl-v10.html
%
\PassOptionsToPackage{hyphens}{url}
\documentclass[a4paper,12pt]{report}
\usepackage{url}
\usepackage{hyperref}
\usepackage{fancyvrb}
\usepackage{enumitem}
\usepackage[T1]{fontenc}
\usepackage{lmodern}
\usepackage{enumitem}
\begin{document}
%
% variables and macros
%
%\defconfidential{~}
\newcommand{\wmq}{IBM MQ }
\newcommand{\wmqz}{\wmq for z/OS }

% Remove the word 'Chapter' 
\renewcommand{\chaptername}{}
% and remove the number
\renewcommand{\thechapter}{}

% turn off chapter and section numbers
\setcounter{secnumdepth}{0}


% starts here
%

\renewcommand{\abstractname}{Release Note}
\begin{abstract}
The original "User Guide" provided with MS15 has been converted to 
Latex from IBM's BookManager format and follows.

The code, mnglogs.exec, was most recently updated in 2014 to handle 
the 8 byte 
"wide" RBA capability provided in \wmq Version 8.0
\end{abstract}

%-------------Add final date at publication ----------------
\date{20th December, 1999}
\title{
MS15: MQSeries for MVS/ESA\\ Archive log housekeeping\\
Version 1.2\\
User Guide
}
\author{Pete Siddall\\
IBM Hursley Development Lab\\
pete\_siddall@uk.ibm.com
}
\maketitle
%:frontm
%:tipage
%:notices

\clearpage
\begin{flushleft}
\begin{tabular}{|p{\textwidth}|}
%\begin{tabular}{| l |}
\hline
Take Note!\\
\hline
Before using this User's Guide and the product it supports, be sure
to read the general information under
"Notices".\\
\hline
\end{tabular}
\end{flushleft}

%
% =====================================================================
%:vnotice.
%\subsection[Notices]{Notices}
%----------------------------------------------------------------------
%
\begin{description}
\item [Fourth Edition, April 2005]
This edition solves some issues with deleting archives.
If the delete of an archive log or bsds copy fails for a reason other
than entry not found, the TSO error messages are shown.

A new option, DELPURGE, allows the PURGE operand of the DELETE
command to be used, overriding any expiration processing.

\item [Third Edition, September 2002]
This edition features changes made to packaging as a result of
re-issuing the software to accomodate changes required to support
WebSphere MQ for z/OS Version 5 Release 3.

\item [Second Edition, December 1999]
This edition applies to Version 1.1 of MQSeries for MVS/ESA
Archive log housekeeping
and to all subsequent releases and modifications until
otherwise indicated in new editions.
\end{description}
%
A form for reader's comments is provided at the back of this publication.
If the form has been removed, address your comments to:
\begin{verbatim}
IBM United Kingdom Laboratories
Transaction Systems Marketing Support (MP102)
Hursley Park
Hursley
Hampshire, SO21 2JN, England
\end{verbatim}

When you send information to IBM, you grant IBM a non-exclusive right
to use or distribute the information in any way it believes appropriate
without incurring any obligation to you.  You may continue to use the
information that you supply.
%.config rcf off
%
\subsection{Copyright}
Copyright \copyright 1999, 2016 IBM Corporation and other Contributors.

All rights reserved. This program and the accompanying materials
are made available under the terms of the Eclipse Public License v1.0
which accompanies this distribution, and is available at
http://www.eclipse.org/legal/epl-v10.html

%:toc.
%.*:figlist.
%.* =====================================================================
\chapter[Notices]{Notices}
%.*----------------------------------------------------------------------
%.*

\textbf{The following paragraph does not apply in any country where such
provisions are inconsistent with local law.
}

\noindent
INTERNATIONAL BUSINESS MACHINES CORPORATION PROVIDES THIS
PUBLICATION "AS IS" WITHOUT WARRANTY OF ANY KIND, EITHER EXPRESS OR
IMPLIED, INCLUDING, BUT NOT LIMITED TO, THE IMPLIED WARRANTIES OF
MERCHANTABILITY OR FITNESS FOR A PARTICULAR PURPOSE.

Some states do not allow disclaimer of express or implied warranties in certain
transactions, therefore this statement may not apply to you.

References in this publication to IBM products, programs, or services do not
imply that IBM intends to make these available in all countries in which IBM
operates.

Any reference to an IBM licensed program or other IBM product in this
publication is not intended to state or imply that only IBM's program or other
product may be used.  Any functionally equivalent program that does not infringe
any of the intellectual property rights may be used instead of the IBM product.
Evaluation and verification of operation in conjunction with other products,
except those expressly designated by IBM, is the user's responsibility.

IBM may have patents or pending patent applications covering subject matter
in this document.  The furnishing of this document does not give you any license
to these patents.  You can send license inquiries, in writing, to the IBM
Director of Licensing, IBM Corporation, 500 Columbus Avenue, Thornwood, New York
10594, USA.

The information contained in this document has not be submitted to any formal
IBM test and is distributed AS IS.  The use of the information or the
implementation of any of these techniques is a customer responsibility and
depends on the customer's ability to evaluate and integrate them into the
customer's operational environment.  While each item has been reviewed by IBM
for accuracy in a specific situation, there is no guarantee that the same or
similar results will be obtained elsewhere.  Customers attempting to adapt these
techniques to their own environments do so at their own risk.

The following terms
are trademarks of the International Business Machines Corporation
in the United States and/or other countries:
\begin{itemize}[noitemsep]
\item MQSeries
\item IBM
\item MVS
\item ESA
\item MQM
\end{itemize}

\begin{description}[labelwidth=3cm]
\item [Date] Changes
\item [April 2005] Description of the new DELPURGE option.
\item [September 2002] The SLEEP module originally supplied as part of this package is now
no longer required as it is available in base OS/390 and z/OS function.
\item [20th December 1999] Added function to enable this tool to be tied into backup procedures
to ensure that archive logs are only deleted once suitable backups have
been taken.
\item [20th October 1999] Initial release
\end{description}
%.config acksoa off
%
%
% :body
%***********************************************************************
\chapter{Introduction}
%***********************************************************************
%

This document describes an exec designed to housekeep archive logs
stored on disk by MQSeries for MVS/ESA, i.e. it will delete redundant
archive logs in order to save disk space.  The archive logs are regarded
as redundant from the MQSeries for MVS/ESA recovery perspective, i.e.
they are not required by the product to recover from a failure.  If you
need to keep archive logs for reasons other than recovery then you
should not use this exec.

You must keep in mind that there are certain recovery scenarios,
where you might need the archive logs, even though they are eligible to
be deleted by the exec, for instance if you need to restart using a
backup of a pageset that has been damaged, you will need the active and
archive log backups that go with the pageset backup in order to
recover your queue manager.  For this reason, it is essential that you
consider use of this exec in association with your backup procedures.

Note: This Supportpac requires SupportPac MA19 in order to submit
commands to the queue manager from a REXX exec.

\section{Installation}
Take the following actions to install the SupportPac from the
MS15.ZIP file: 
\begin{enumerate}
\item
Use \textbf{INFOZIP's} UNZIP to unpack the \textbf{MS15.ZIP} file.

This produces
\begin{description}
\item [ms15.src] Source library
\item [ms15.loa] Load library
\item [license2.txt] License file
\end{description}

\item
The library files need to be transferred to the destination TSO system
as sequential binary filed with a record format of FB 80.
Use one of the following methods to accomplish this:
\begin{itemize}
% :li.Use the Communications Manager/2
% :hp4.SEND:ehp4. commands below to send the files to TSO
% as a sequential binary files\colon 
% :lines.
%
% \textbf{send ms15.src A\colon ms15.srcseq}
%
% \textbf{send ms15.loa A\colon ms15.loadseq}
% :elines.
% where \textbf{A} is the TSO session ID.
\item To send them via ftp ensure the BINARY option is set then use
the following commands:
\begin{verbatim}
 site fixrecfm 80
 put ms15.src ms15.srcseq
 put ms15.loa ms15.loadseq
\end{verbatim}
\item With Personal Communications, use the "Send Files to Host"
option under the Transfer menu item to transmit to TSO
\begin{verbatim}
PC File             ms15.src etc
Host File           ms15.srcseq etc
Transfer Type       loadlib
\end{verbatim}
The Transfer type of \textbf{loadlib} may need to be correctly
setup. To do this, use the "Setup.Define Transfer Types" option
under the Transfer menu item and create the \textbf{loadlib} type
with the Ascii, CRLF and Append checkboxes all unselected, the
\textbf{Fixed} radio button selected and the \textbf{LRECL} set to
\textbf{80}.
\end{itemize}
\item
On TSO, issue the following commands to unload these sequential files
into TSO partitioned datasets:
\begin{itemize}
\item
\begin{verbatim}
receive indsname(ms15.srcseq)
\end{verbatim}
when prompted for a filename, reply
\begin{verbatim}
dsn(ms15.source)
\end{verbatim}
\item
\begin{verbatim}
receive indsname(ms15.loadseq)
\end{verbatim}
when prompted for a filename, reply
\begin{verbatim}
dsn(ms15.load)
\end{verbatim}
\end{itemize}
\item Confirm that the datasets
have been unloaded correctly by browsing the PDS and
ensuring the appropriate members exist as shown in the following table:\\
\begin{tabular}{| l | l | l |}
\hline
%:table cols='* * *' split=no rules=vert
%:thd.
\textbf{DATASET} & \textbf{MS15 LOAD} & \textbf{MS15 SOURCE}\\
\hline
%:ethd.
%:row.
MEMBERS & SLEEP & MNGLOGS\\
%:row
 & & MNGLOGSJ\\
%:row
 & & SLEEP\\
%:row
 & & SLEEPJ\\
\hline
\end{tabular}
\end{enumerate}
%
%***********************************************************************
\subsection{Enabling the exec}
%***********************************************************************
%
To run the exec you should tailor the MNGLOGSJ job to your
requirements, it looks like this: 
\begin{Verbatim}[commandchars=\\\{\}]
//<jobcard>
// SET MA19LOAD=\textbf{<MA19.LOADLIB>}
// SET MS15=\textbf{<MS15 HIGH LEVEL QUALIFIERS>}
// SET MQHLQS=\textbf{<MQSERIES HIGH LEVEL QUALIFIERS>}
//RMNGLOGS EXEC PGM=IKJEFT01,REGION=2M,DYNAMNBR=99
//SYSEXEC  DD DSN=&MS15..SOURCE,DISP=SHR
//SYSTSPRT DD SYSOUT=*
//SYSPRINT DD SYSOUT=*
//STEPLIB  DD DSN=&MA19LOAD.,DISP=SHR           /* MA19 SUPPORTPAC */
//         DD DSN=&MQHLQS..SCSQANLE,DISP=SHR
//         DD DSN=&MQHLQS..SCSQAUTH,DISP=SHR
//SYSTSIN  DD *
 \%MNGLOGS QMGR(\textbf{<QUEUE MANAGER NAME E.G. VCMA>}) -
   HLQS(\textbf{<MQSERIES HIGH LEVEL QUALIFIERS E.G. MQ.V210>}) -
   BS1(\textbf{<BOOTSTRAP DATA SET NO.1 NAME E.G. MQ.VCMA.BSDS01>}) -
   BS2(\textbf{<BOOTSTRAP DATA SET NO.2 NAME E.G. MQ.VCMA.BSDS02>}) -
   BSDS(\textbf{<DUAL BOOTSTRAPS? Y OR N>}) -
   SLEEP(\textbf{<SLEEP PERIOD BETWEEN CHECKS IN SECONDS>})
   BACKUP(\textbf{<TAKE BACKUPS? Y OR N>}) -
   RUNTYPE(\textbf{<O - ONCE, C - CONTINUOUS>}) -
   RESTARTRBA(\textbf{<RESTART RELATIVE BYTE ADDRESS>}) -
   DELPURGE(\textbf{<USE DELETE PURGE? Y or N>})
/*
\end{Verbatim}

You will need one copy of this job for each queue manager you wish to
run it against.
\begin{enumerate}
\item Replace <jobcard> with valid job details for your system
\item Replace <MA19.LOADLIB> with the fully qualified name of the MA19
SupportPac load library.
\item Replace <MS15 HIGH LEVEL QUALIFIERS> with the high level qualifiers
of the MS15 SupportPac data sets.
\item Replace <MQSERIES HIGH LEVEL QUALIFIERS> with the high level
qualifiers of the MQSeries for MVS/ESA data sets.
\item Replace <QUEUE MANAGER NAME E.G. VCMA> with the name of the queue
manager whose archive logs the exec will be monitoring.
\item Replace <MQSERIES HIGH LEVEL QUALIFIERS E.G. MQ.V210> with the high
level qualifiers of the MQSeries for MVS/ESA data sets.
\item Replace <BOOTSTRAP DATA SET NO.1 NAME E.G. MQ.VCMA.BSDS01> with the
name of your 1st bootstrap data set.
\item If you have a 2nd bootstrap data set, then replace
<BOOTSTRAP DATA SET NO.2 NAME E.G. MQ.VCMA.BSDS02> with the name of
your 2nd bootstrap data set, otherwise delete this particular record fro
the job.
\item If you are using dual bootstrap data sets for this queue manager,
then replace <DUAL BOOTSTRAPS? Y OR N> with Y, otherwise replace it with
N.  It is essential that this parameter is set correctly otherwise it
could result in dual bootstrap data sets becoming inconsistent.
\item Replace <SLEEP PERIOD BETWEEN CHECKS IN SECONDS> with the time in
seconds that you want the exec to sleep between checking the status of
the archive logs, for example, 300 will cause it to sleep for 5 minutes.
\item If you want the exec to start your backup procedures, you need to do
the following:
\begin{enumerate}
\item Edit MNGLOGS EXEC, and change the perform\_backups routine at the
very bottom of the exec to run your backup procedures.  The backups
should include:
\begin{itemize}
\item Pagesets
\item Bootstrap datasets
\item Active logs
\item Archive logs
\item Object definitions if you need to keep them independently.
\end{itemize}

To drive the backups, the exec uses two variables to hold the restart
relative byte address: 

\begin{description}
\item [Variable] Content
\item [restart\_rba] The restart RBA in hexadecimal.
\item [decimal\_restart\_rba] The restart RBA in decimal.
\end{description}

\item In MNGLOGSJ replace <TAKE BACKUPS? Y OR N> with Y.
\end{enumerate}
\item If you do not want MNGLOGS to start your backup procedures, replace
<TAKE BACKUPS? Y OR N> with N.
\item If you want the exec to check the archive logs once only i.e. check
and delete any redundant archive logs, and then terminate, replace
<O - ONCE, C - CONTINUOUS> with O.  If however you want the exec to
run continuously in parallel with the queue manager then replace
<O - ONCE, C - CONTINUOUS>with C.
\item If you want to specify to MNGLOGS the restart RBA you want it to
start with then replace <RESTART RELATIVE BYTE ADDRESS> with the restart
RBA you want it to use in hexadecimal notation, for example:
6DF37FFFF.  This is an alternative way for you to tie the exec into your
backup procedures.  If you do not want to use this parameter, remove it
from the JCL, and remove the continuation mark '-' from the end of the
previous line.
\item If you want to delete archives, regardless of the retention period
specified for the dataset (ARCRETN in CSQ6ARVP for ZPARMS).  Use the
operand DELPURGE(Y), so that a DELETE dsn PURGE command is used.
\end{enumerate}
%***********************************************************************
\subsection{SLEEP function}
%***********************************************************************
%
The USS services sleep() function is now used instead of the SLEEP
module described in the following paragraph.

A TSO sleep function is provided with this SupportPac, which enables
the exec to sleep (i.e. Not use any system resources) for a specified
period.  The source for this function (SLEEP) and sample JCL to build
it (SLEEPJ) can be found in the MS15.SOURCE data set.
%***********************************************************************
\chapter{What it does}
%***********************************************************************
The exec is intended to manage archive logs on disk so that you only
keep the ones MQSeries needs to be able to recover from a failure or
backout incomplete work, however there are certain disaster scenarios,
for instance pageset damage, which might require you to recover from
archive logs which are no longer necessary to recover a clean system.
It is therefore vital that you consider backing up your archive logs
along with your pagesets, bootstrap data sets, and active logs, before
deleting them from your main storage, to this end the exec has been
written to allow you to tie it in with your backup procedures.

The exec can be run in one of two modes:
\begin{enumerate}
\item Once only:

Delete any redundant archive logs and then stop.
\item Continuous execution:

For as long as the queue manager is running, every so many seconds,
delete any redundant archive logs and then sleep.This allows you to
keep mainline DASD usage to the minimum required by the queue manager.
\end{enumerate}

You can tie the exec into your backup procedures in one of two
ways:
\begin{enumerate}
\item By specifying as a parameter the Restart Relative Byte Address (RBA)
you have used for your backups, and running the exec in the "Once only"
mode.
\item By specifying the backup parameter as "Y".  This will cause the
exec to run a subroutine called perform\_backups.  You should tailor
this subroutine to execute your backup procedures.
\end{enumerate}

The exec performs the following processing:
\begin{enumerate}
\item Determines whether the queue manager is active, and if it is:
\begin{enumerate}
\item Using either the MQSeries DISPLAY USAGE command, or the RESTARTRBA
parameter, determines the restart RBA
(Relative Byte Address) to be used to determine which archive logs can
be deleted.
\item If requested, it runs the routine which can be tailored to run your
backup procedures.
\item Using the CSQJU004 utility shipped with MQSeries for MVS/ESA,
it determines what archive logs the queue manager has produced.
\item It Determines which of the archive logs are out of date with respect
to the restart RBA and deletes them.  DELETE PURGE may be used to
override dataset expiration processing, if DELPURGE(Y) parameter has
been specified.
\item The DELETE is recorded as successful if either, it completes with
return code zero, or, it only fails because the entry is not found (it
has already been deleted).  In other situations, the error messages
resulting from the failed delete are printed.

:note.It is not possible at this point in time to remove the references
to the deleted archive logs from the bootstrap data sets as MQSeries has
an exclusive lock on the BSDS.
\item It then goes to sleep for the specified period.
\item Repeats the above steps, unless you have specified that it should
only perform them once.
\end{enumerate}
\item When the queue manager is found to be no longer active and if
there have been successful deletions, then
the
CSQJU003 utility is used to update the bootstrap data sets, removing the
references to the deleted archive logs from them.
\end{enumerate}
\section{Possible Extensions}
The process\_archives routine could be extended to use the LISTDSI
function to determine the current location of archive logs.
Once established, the routine could use \verb!HMIGRATE! to cause
HSM to migrate newly created archive logs, or \verb!HDELETE! instead
of \verb!DELETE! to delete old logs without causing them to be
recalled first.
\end{document}
